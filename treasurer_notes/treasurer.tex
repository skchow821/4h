\documentclass{article}

\title{Treasurer - 4H}
\author{Nimit Singhania}

\begin{document}

\maketitle

The role of treasurer is an important one in any club or community. The treasurer accounts for how the money is spent by various projects within the club, and ensure that it meets the incoming funds and available resources. The treasurer needs to track the movement of resources, including items purchased for the projects, and the amount spent or incurred for the same. This can seem a daunting job, however using proper documentation and tabulation, one can quickly and easily account for all the money spent and used by the club projects. The treasurer not just helps free other people from having to think about how the resources are used, but they gain an important skill of managing and tracking information which is useful in any task performed by them in future.

Organising and managing information is a difficult task and can be learnt by experience. The position of treasurer would enable the youth to learn this important skill by experience. Some handy tips to get started on managing information is here:
\begin{enumerate}
\item Start with only what is required. Focus on short term requirements and only use resources/documents that are needed for these goals.
\item Iterate upon your plans based on requirement. If you feel your current organization is not sufficient for managing all the activity that needs to be performed, get help with more organization and resources.
\item Act independently. Listen to what everyone says, but act on your own accord. This will help build judgement and knack to have everyone's input.
\item Focus on your responsibility. You are accountable for the role you are assigned. While it is okay to do other things and participate in other activities, fulfilling your role and duties assigned will help successfully run the club and allow smooth functioning of club activities.
\end{enumerate}

\end{document}
